\documentclass{report}
\usepackage{tikz}
\usetikzlibrary{arrows,automata}
\usepackage{amsmath}
\usepackage{amssymb}
\setlength{\parindent}{0cm}
\setlength{\parskip}{0.75em}
%%
\usepackage{ifthen}
\usepackage[a4paper, total={6in, 8in}]{geometry}
\usepackage{url}
\usepackage{listings}

%% caption less table to specify restful primitives
\newcommand{\restful}[1]
{
\begin{center}
\begin{tabular}{l p{12cm}}
\hline
#1
\hline
\end{tabular}
\end{center}
\vspace{6pt}
}


%% caption less table to specify restful uri
\newcommand{\routes}[1]
{
\begin{center}
\begin{tabular}{p{\textwidth}}
\hline
#1
\hline
\end{tabular}
\end{center}
\vspace{6pt}
}
%
%
%
  
\newcommand{\mimetype}[2]{#1 & #2 \\\noalign{\smallskip}}
\newcommand{\head}[2]{#1 & #2 \\\noalign{\smallskip}}
\newcommand{\uri}[2]{\url{#1} \\ #2 \vspace{8pt} \\\noalign{\smallskip}}
\newcommand{\env}[2]{#1 & #2 \\\noalign{\smallskip}}
%

%%% resource definition table
\newcommand{\resource}[1]
{
\begin{center}
\begin{tabular}{l l p{12cm}}
\hline
#1
\hline
\end{tabular}
\end{center}
\vspace{6pt}
}
\newcommand{\attr}[3]{{\tt #1} & #2 & #3 \\\noalign{\smallskip}}

%
%
\newcommand{\request}[6]
{
\begin{center}
\begin{tabular}{l p{12cm}}
\hline
Synopsis:  & #1 \url{#2} \\\noalign{\smallskip}
           & #3     \vspace{4pt}\\ 
Precondition:  & #4 \vspace{4pt}\\
Postcondition: & #5 \vspace{4pt}\\
Status code:   & #6 \\
\hline
\end{tabular}
\end{center}
\vspace{6pt}
}

\newcommand{\sep}{\\\noalign{\smallskip} &}

\newcommand{\status}[1]
{
\ifthenelse{\equal{#1}{100}}{100 Continue}{}%
\ifthenelse{\equal{#1}{101}}{101 Switching Protocols}{}%
\ifthenelse{\equal{#1}{200}}{200 OK}{}%
\ifthenelse{\equal{#1}{201}}{201 Created}{}%
\ifthenelse{\equal{#1}{202}}{202 Accepted}{}%
\ifthenelse{\equal{#1}{203}}{203 Non-Authoritative Information}{}%
\ifthenelse{\equal{#1}{204}}{204 No Content}{}%
\ifthenelse{\equal{#1}{205}}{205 Reset Content}{}%
\ifthenelse{\equal{#1}{206}}{206 Partial Content}{}%
\ifthenelse{\equal{#1}{300}}{300 Multiple Choices}{}%
\ifthenelse{\equal{#1}{301}}{301 Moved Permanently}{}%
\ifthenelse{\equal{#1}{302}}{302 Found}{}%
\ifthenelse{\equal{#1}{303}}{303 See Other}{}%
\ifthenelse{\equal{#1}{304}}{304 Not Modified}{}%
\ifthenelse{\equal{#1}{307}}{307 Temporary Redirect}{}%
\ifthenelse{\equal{#1}{302}}{302 Found}{}%
\ifthenelse{\equal{#1}{400}}{400 Bad Request}{}%
\ifthenelse{\equal{#1}{401}}{401 Unauthorized}{}%
\ifthenelse{\equal{#1}{402}}{402 Payment Required}{}%
\ifthenelse{\equal{#1}{403}}{403 Forbidden}{}%
\ifthenelse{\equal{#1}{404}}{404 Not Found}{}%
\ifthenelse{\equal{#1}{405}}{405 Method Not Allowed}{}%
\ifthenelse{\equal{#1}{406}}{406 Not Acceptable}{}%
\ifthenelse{\equal{#1}{407}}{407 Proxy Authentication Required}{}%
\ifthenelse{\equal{#1}{408}}{408 Request Timeout}{}%
\ifthenelse{\equal{#1}{409}}{409 Conflict}{}%
\ifthenelse{\equal{#1}{410}}{410 Gone}{}%
\ifthenelse{\equal{#1}{411}}{411 Length Required}{}%
\ifthenelse{\equal{#1}{412}}{412 Precondition Failed}{}%
\ifthenelse{\equal{#1}{413}}{413 Request Entity Too Large}{}%
\ifthenelse{\equal{#1}{414}}{414 Request-URI Too Long}{}%
\ifthenelse{\equal{#1}{415}}{415 Unsupported Media Type}{}%
\ifthenelse{\equal{#1}{416}}{416 Requested Range Not Satisfiable}{}%
\ifthenelse{\equal{#1}{417}}{417 Expectation Failed}{}%
\ifthenelse{\equal{#1}{422}}{422 Unprocessable Entity}{}%
\ifthenelse{\equal{#1}{500}}{500 Internal Server Error}{}%
\ifthenelse{\equal{#1}{501}}{501 Not Implemented}{}%
\ifthenelse{\equal{#1}{502}}{502 Bad Gateway}{}%
\ifthenelse{\equal{#1}{503}}{503 Service Unavailable}{}%
\ifthenelse{\equal{#1}{504}}{504 Gateway Timeout}{}%
\ifthenelse{\equal{#1}{505}}{505 HTTP Version Not Supported}{}%
}

\newcommand{\httpcode}[2]{\status{#1} & #2 \\\noalign{\smallskip}}

\newcommand{\example}[1]{\noindent {\bf #1}}





\begin{document}

\title{xxx}
\author{}
\maketitle

%%%%%%%%%%%%%%%%%%%%%%%%%%%%%%%%%%%%%%%%%%%%%%%%%%%%%%%%%%%%%%%%%%%%%%%%%%%%%%%%
%%
%%  
%%
%%%%%%%%%%%%%%%%%%%%%%%%%%%%%%%%%%%%%%%%%%%%%%%%%%%%%%%%%%%%%%%%%%%%%%%%%%%%%%%%
\chapter{Introduction}

%%
%% short description about the service.
\section{Rationale}
tbd.

%%
%%
\section{Terms of Service}
tbd.

%%
%%
\section{Environments}

\restful{
   \env{dev}{xxx}
   \env{stage}{xxx}   
   \env{live}{xxx}
}


\section{Common behavior}

%%
%% list of media types consumed / produces by service 
\subsection{Media Types}

\subsubsection{Consumes}
\restful{
   \mimetype{application/ld+json} 
            {tbd}
}

\subsubsection{Produces}
\restful{
   \mimetype{application/json} 
            {tbd}
}


\subsection{Protocol}

%%
%% service protocols, transfer requirements, etc.
\subsubsection{Schema}
tbd.

%%
%% list of common request parameters (e.g. headers)
\subsubsection{Request}
\restful{
   \head{Accept} 
        {tbd}
        
   \head{Content-Type} 
        {tbd}
}

%%
%% list of common response parameters
\subsubsection{Response}
\restful{
   \head{Content-Type} 
        {tbd.}
}

%%
%% list of plausible response codes 
\subsubsection{Status}
\restful{
   \httpcode{200} {operation successfully completed}
   \httpcode{201} {operation successfully completed, and resource was created}
   \httpcode{406} {the supplied request does not conform API specification, requested attributes are not known}
   \httpcode{500} {for any abnormally terminated requests}
}

%%
%% api authentication schema
\subsection{Authentication}
void




%%%%%%%%%%%%%%%%%%%%%%%%%%%%%%%%%%%%%%%%%%%%%%%%%%%%%%%%%%%%%%%%%%%%%%%%%%%%%%%%
%%
%%  
%%
%%%%%%%%%%%%%%%%%%%%%%%%%%%%%%%%%%%%%%%%%%%%%%%%%%%%%%%%%%%%%%%%%%%%%%%%%%%%%%%%
\chapter{Resource model}

\section{Rationale}
tbd

\section{Definition}

%%
%% a definition of the resource (domain objects), 
%% specifies request / response structures
\subsection{TBD}
tbd.

\resource{
   \attr{@id}{string}{tbd}
   \attr{@type}{string}{tbd}
   \attr{title}{string}{tbd}
}

\paragraph{example} inline example

\subsubsection{example}
\begin{verbatim}
code example
\end{verbatim}


%%%%%%%%%%%%%%%%%%%%%%%%%%%%%%%%%%%%%%%%%%%%%%%%%%%%%%%%%%%%%%%%%%%%%%%%%%%%%%%%
%%
%%  
%%
%%%%%%%%%%%%%%%%%%%%%%%%%%%%%%%%%%%%%%%%%%%%%%%%%%%%%%%%%%%%%%%%%%%%%%%%%%%%%%%%
\chapter{Interface}

%%
%% interface summary including end-point tempates 
\section{Summary}
tbd.

\routes{
   \uri{/collection/:abc/resource/:def} 
       {tbd}
}

%%
%% detailed resource specification
\section{End-points}

%%
%% end-point specification 
\subsection{/collection/:abc/resource/:def}
tbd.

%%
%% method specification
%%  - method id
%%  - path template
%%  - description
%%  - preconditions, parameters, etc (use \sep to build list)
%%  - postcondition (use \sep to build list)
%%  - expected status codes
\request{GET} {/collection/:abc/resource/:def}
   {tbd}
   {Accept: application/json \sep q = string}
   {Content-Type: application/json}
   {\status{200} \sep \status{404} \sep \status{500}}
  
\example{example}
\begin{verbatim}
curl -v -X GET http://localhost:8080/collection/x/resource/y
* About to connect() to localhost port 8080 (#0)
*   Trying 127.0.0.1... connected
* Connected to localhost (127.0.0.1) port 8080 (#0)
> GET /collection/x/resource/y HTTP/1.1
> User-Agent: curl/7.21.4 (universal-apple-darwin11.0) libcurl/7.21.4 OpenSSL/0.9.8r zlib/1.2.5
> Host: localhost:8080
> Accept: */*
> 
< HTTP/1.1 200 OK
< Server: MochiWeb/1.0 (Any of you quaids got a smint?)
< Date: Fri, 01 Jun 2012 20:01:36 GMT
< Content-Length: 175
< Conten-Type: application/json
< 
< []
* Connection #0 to host localhost left intact
* Closing connection #0
\end{verbatim}

   
\end{document}
